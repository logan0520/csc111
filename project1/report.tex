\documentclass[11pt]{article}
\usepackage{amsmath}
\usepackage{amsfonts}
\usepackage{amsthm}
\usepackage[utf8]{inputenc}
\usepackage[margin=0.75in]{geometry}

\title{CSC111 Winter 2026 Project 1}
\author{Sanghyun (Logan) Choo, Mupin (Jay) An}
\date{\today}

\begin{document}
\maketitle

\section*{Running the game}
We are able to run our game by simply running \texttt{adventure.py}.

\section*{Game Map}
Example game map below (edit it to show your actual game map):

\begin{verbatim}
1   2  -1   4
5  -1  -1   8
9  10  11  12
13  14  15  16
\end{verbatim}

\textbf{Location details:}
\begin{itemize}
    \item Location 1: Front Campus (Starting location)
    \item Location 2: University College
    \item Location 4: Second Cup (T-card, Lucky Mug)
    \item Location 5: McLennan Labs
    \item Location 8: Woodsworth College
    \item Location 9: Trinity College
    \item Location 10: New College
    \item Location 11: Bahen Centre (USB Drive, Laptop Charger)
    \item Location 12: Earth Sciences Centre
    \item Location 13: Robarts Library (Dorm Key) - \textbf{LOCKED: Requires T-card}
    \item Location 14: Sidney Smith Hall
    \item Location 15: Oak House - \textbf{LOCKED: Requires Dorm Key, DEPOSIT location}
    \item Location 16: Athletic Centre
\end{itemize}

Starting location is: 1: Front Campus

\section*{Game solution}
The best solution requires 30 commands to win the game. The strategy is picking up the T-card to unlock Robarts Library, and then get the Dorm Key, and then picking up and depositing all three required items at the Oak House.
List of commands:
\begin{verbatim}
go south
go south
go east
go east
go east
go north
go north
pick up lucky mug
pick up t-card
go south
go south
go west
go west
go south
drop lucky mug
go west
pick up dorm key
go east
drop t-card
pick up lucky mug
go east
deposit lucky mug
go north
pick up laptop charger
go south
deposit laptop charger
go north
pick up usb drive
go south
deposit usb drive
\end{verbatim}

\textbf{Total moves:} 21 moves

\textbf{Final score:} 185 points

\section*{Lose condition(s)}
Description of how to lose the game: If the player uses all the 50 moves then they lose the game.

List of commands:
\begin{verbatim}
go south
go north
go south
go north
go south
go north
go south
go north
go south
go north
go south
go north
go south
go north
go south
go north
go south
go north
go south
go north
go south
go north
go south
go north
go south
go north
go south
go north
go south
go north
go south
go north
go south
go north
go south
go north
go south
go north
go south
go north
go south
go north
go south
go north
go south
go north
go south
go north
go south
go north
\end{verbatim}

Which parts of your code are involved in this functionality:

\textbf{Code locations:}
\begin{itemize}
    \item \texttt{adventure.py}, lines 409: Check that user made over 50 moves.
    \item \texttt{adventure.py}, lines 381-394: Moves count increment \texttt{game.moves}
\end{itemize}

% Copy-paste the above if you have multiple lose conditions and describe each possible way to lose the game

\section*{Inventory}

\begin{enumerate}
\item All location Ids that involve items in the game:
\begin{itemize}
    \item Location 4 (Second Cup): T-card, Lucky Mug
    \item Location 11 (Bahen Centre): USB Drive, Laptop Charger
    \item Location 13 (Robarts Library): Dorm Key
    \item Location 15 (Oak House): Target deposit location for all three required items
\end{itemize}
\item Item data:
\begin{enumerate}
    \item For Item 1:
    \begin{itemize}
    \item Item name: USB Drive
    \item Item start location Id: 11
    \item Item target location Id: 15
    \item Target points: 50
    \end{itemize}
    
    \item For Item 2:
    \begin{itemize}
    \item Item name: Laptop Charger
    \item Item start location Id: 11
    \item Item target location Id: 15
    \item Target points: 75
    \end{itemize}
    
    \item For Item 3:
    \begin{itemize}
    \item Item name: Lucky Mug
    \item Item start location Id: 4
    \item Item target location Id: 15
    \item Target points: 60
    \end{itemize}
    
    \item For Item 4:
    \begin{itemize}
    \item Item name: T-card
    \item Item start location Id: 4
    \item Item target location Id: 13
    \item Target points: 0 (item for unlocking Robarts Library)
    \end{itemize}
    
    \item For Item 5:
    \begin{itemize}
    \item Item name: Dorm Key
    \item Item start location Id: 13
    \item Item target location Id: 15
    \item Target points: 0 (item for unlocking Oak House)
    \end{itemize}
\end{enumerate}

List of commands:
\begin{verbatim}
go south
go south
go east
go east
pick up usb drive
inventory
\end{verbatim}

This demonstrates picking up the USB Drive at the Bahen Centre (Location 11) and then checking the inventory to see if the inventory is carrying 1 out of 2 items.
    \item Exact command(s) that should be used to pick up an item (choose any one or more items for this example), and the command(s) used to use/drop the item (can copy the list you assigned to \texttt{inventory\_demo} in the \texttt{simulation.py} file)
    \item Which parts of your code (file, class, function/method) are involved in handling the \texttt{inventory} command:
    \begin{itemize}
    \item \texttt{adventure.py}, lines 84-87: \texttt{AdventureGame.\_\_init\_\_()} initializes \texttt{self.score = 0}, \texttt{self.moves = 0}, \texttt{self.inventory = \{\}}, and \texttt{self.deposited\_items = set()}
    \item \texttt{adventure.py}, lines 289-292: Handles \texttt{"inventory"} command
    \item \texttt{adventure.py}, lines 308-328: Pick up function checks if inventory is full (\texttt{len(game.inventory) >= 2}) at lines 315-316
    \item \texttt{simulation.py}, lines 77-96: \texttt{\_pick\_up()} helper code
\end{itemize}
\end{enumerate}

\section*{Score}
\begin{enumerate}

    \item Briefly describe the way players can earn score in your game. Include the first location in which they can increase their score, and the exact list of command(s) leading up to the score increase:

Players earn score by depositing the three required items (USB Drive, Laptop Charger, Lucky Mug) at Oak House (Location 15). Each item has a specific point value:
\begin{itemize}
    \item USB Drive: 50 points
    \item Laptop Charger: 75 points
    \item Lucky Mug: 60 points
\end{itemize}
The first location where a score increase can occur is \textbf{Location 15 (Oak House)}, since items can only be deposited there. The maximum score is 185 points (50 + 75 + 60).

    \item Copy the list you assigned to \texttt{scores\_demo} in the \texttt{simulation.py} file into this section of the report:

\begin{verbatim}
go south
go south
go east
go east
go east
go north
go north
pick up lucky mug
pick up t-card
go south
go south
go west
go west
go south
drop lucky mug
go west
pick up dorm key
go east
drop t-card
pick up lucky mug
go east
deposit lucky mug
\end{verbatim}

This demonstrates earning 60 points by depositing the Lucky Mug at Oak House.

    \item Which parts of your code (file, class, function/method) are involved in handling the \texttt{score} functionality:

\begin{itemize}
    \item \texttt{adventure.py}, line 46: \texttt{AdventureGame} class defines \texttt{score} instance attribute
    \item \texttt{adventure.py}, line 84: \texttt{AdventureGame.\_\_init\_\_()} initializes \texttt{self.score = 0}
    \item \texttt{adventure.py}, lines 293-296: The \texttt{"score"} menu command displays current score, moves, and deposited item count
    \item \texttt{adventure.py}, lines 352-377: The \texttt{"deposit"} command logic calculates points from \texttt{item.target\_points} and adds to \texttt{game.score}
    \item \texttt{game\_entities.py}: \texttt{Item} class stores \texttt{target\_points} attribute for each item
\end{itemize}

\end{enumerate}

\section*{Enhancements}
\begin{enumerate}
    \item Describe your enhancement \#1 here
    \begin{itemize}
        \item Brief description of what the enhancement is (if it's a puzzle, also describe what steps the player must take to solve it): Players have two ways to remove items from their inventory: \texttt{drop} and \texttt{deposit}. After player pick it up the item and then drop it, then it will be just dropped at that place, but to win the game, the player must deposit the item at the target location, which is Oak House (Location 15)
        \item Complexity level (choose from low/medium/high): Medium
        \item Reasons you believe this is the complexity level (e.g., mention implementation details, how much code did you have to add/change from the baseline, what challenges did you face, etc.): Implementing two separate command handler with different logic was challenging. The \texttt{drop} command needed to update both the inventory and the location's item list, while the \texttt{deposit} command needed to check the current location (it must be at Oak House). Also the \texttt{get\_available\_actions()} function show drop and deposit options based on current inventory.
        \item Name the parts of the code which are involved in this enhancement: 
        \begin{itemize}
            \item \texttt{adventure.py}, lines 330-350: Drop command code
            \item \texttt{adventure.py}, lines 352-379: Deposit command code
            \item \texttt{adventure.py}, lines 146-178: \texttt{get\_available\_actions()} function filters available drop and deposit commands
            \item \texttt{simulation.py}, lines 98-114: \texttt{\_drop()} helper method
            \item \texttt{simulation.py}, lines 116-139: \texttt{\_deposit()} helper method
        \end{itemize}
        \item Copy the list you assigned to \texttt{enhancements\_demo} in the \texttt{simulation.py} file into this section of the report:
\begin{verbatim}
go south
go south
go east
go east
go east
go north
go north
pick up lucky mug
pick up t-card
go south
go south
go west
go west
go south
drop lucky mug
go west
pick up dorm key
go east
drop t-card
pick up lucky mug
go east
deposit lucky mug
\end{verbatim}
    \end{itemize}

    % Uncomment below section if you have more enhancements; copy-paste as needed
    \item \textbf{Locked Locations with Key Items}
    \begin{itemize}
    \item Basic description of what the enhancement is: Two locations are locked and require specific key items to enter. Robarts Library (Location 13) requires the T-card, and Oak House (Location 15) requires the Dorm Key. So if the player tries to enter these locations without the required key, then they are not able to enter it. While if they have a key inside of their inventory, then they are able to enter it.
    \item Complexity level (low/medium/high): Medium
    \item Reasons you believe this is the complexity level (e.g., mention implementation details): It requires to check for key items before allowing entry to the locked locations. The game displays lock status messages when the player arrives at the locked location  (Locked or UnLocked). Also the code prevent that the player cannot drop the key at the specific locations where they would be locked in.
    \item Name the parts of the code which are involved in this enhancement:
        \begin{itemize}
            \item \texttt{adventure.py}, lines 224-228: Display LOCKED/UNLOCKED status in location description
            \item \texttt{adventure.py}, lines 231-240: Show lock status messages when at locked locations
            \item \texttt{adventure.py}, lines 337-340: Prevent dropping key at the locations where they would be locked in
            \item \texttt{adventure.py}, lines 381-394: Check for key items before allowing entry to locked locations
            \item \texttt{simulation.py}, lines 141-159: \texttt{\_go()} helper method with lock checking logic
        \end{itemize}
    \item Copy the list you assigned to \texttt{enhancements\_demo} in the \texttt{simulation.py} file into this section of the report:
\begin{verbatim}
go south
go south
go east
go east
go east
go north
go north
pick up lucky mug
pick up t-card
go south
go south
go south
go west
go north
go west
go west
go south
drop lucky mug
go west
pick up dorm key
go east
go east
\end{verbatim}
    \end{itemize}
\end{enumerate}


\end{document}
